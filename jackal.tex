\documentclass[]{clearpath-manual}
\graphicspath{{gen/}}

\begin{document}

\manualcover{cover-page.pdf}
\tableofcontents

\section{Introduction}

Jackal is a rugged, lightweight, fast, and easy-to-use unmanned ground vehicle for ROS
Indigo, presented by Clearpath Robotics.

Jackal includes a standard internal PC, as well as basic IMU and GPS. Standard
perception modules are available, including URDF and simulator integration, and
demonstration applications.

Please inquire with Clearpath Robotics for details. See \nameref{contact} on page
\pageref{contact} for contact information.

\subsection{What's Included}

Contained in your Jackal shipment are the following items:

\begin{itemize}[nolistsep]
  \item Jackal UGV
  \item 270 watt-hour lithium battery pack
  \item 110V/220V universal charger
  \item Sony Bluetooth controller
  \item Jackal User Manual
\end{itemize}

If you elected to purchase standard payload modules or custom integration services with
Jackal, then additional equipment will be included per your specific configuration, plus
further documentation as required.

\subsection{Hardware Overview}

Jackal's external features include the mounting pattern on the lid panel, \SI{190}{\mm} diameter
wheels, human machine interface panel (HMI), and lid panel latches. The HMI panel is shown in
\autoref{hmi}, and includes from left: motor button, comms indictor, wifi indicator, battery
indicator, and system power button.

% TODO: Labeled drawing of external features.

\begin{figure}[ht]
  \centering
  \includegraphics[width=8.0cm]{hmi.pdf}
  \caption{HMI panel.}
  \label{hmi}
\end{figure}

\begin{warning}[Incomplete Feature]
Note that as of the 0.1 firmware, Jackal's HMI indicators and motor enable/disable button are
non-functional. Please expect an update in the near future to correct this.
\end{warning}

\begin{figure}[pt]
  \centering
  \def\svgwidth{12cm}
  \input{gen/battery-view.pdf_tex}
  \caption{Battery area inside Jackal.}
  \label{int}
\end{figure}

\begin{figure}[pb]
  \centering
  \def\svgwidth{14cm}
  \input{gen/user-tray.pdf_tex}
  \caption{Computer and user tray.}
  \label{tray}
\end{figure}

To access Jackal's interior, actuate the latches under the front end of the lid, on the opposite end from
the HMI. When you lift the lid, you will see Jackal's onboard Li-Ion battery pack, and its two connectors.
The large Anderson Power Pole connector is to supply power to Jackal and must be connected in order for
Jackal to operate. The smaller white Molex connector allows the battery pack to be charged inside Jackal 
while Jackal is powered off. It is recommended to connect both. The interior components of Jackal are
labeled in \autoref{int}.

Finally, you may undo the thumbscrews which hold Jackal's computer tray to the lid. The tray lowers,
revealing Jackal's onboard Mini-ITX PC, user power supplies, and internal user hardware mounting area.
Please see \autoref{tray} and \autoref{upb} for the components of the tray and user power supplies.
Note that the fused user power is available as four-pin Molex connectors, or a plug-in screw terminal
block. For more information on integrating payloads electrically, see \autoref{payload-elec},
\nameref{payload-elec}.

\begin{figure}[hb]
  \centering
  \def\svgwidth{15cm}
  \input{gen/user-power.pdf_tex}
  \caption{User power supply.}
  \label{upb}
\end{figure}

\subsection{System Architecture}

Like many ROS robots, Jackal is built around an x86 PC running Ubuntu, paired with a
32-bit MCU. The MCU handles IO, power supply monitoring, and motor control, as well as
supplying data from the integrated IMU and GPS receiver. The communication channel
between the MCU and PC is a Full Speed USB connection, with the MCU operating as a
standard serial CDC device.

The communication protocol used is rosserial. An instance of the \lstinline{rosserial_server}
serial node is embedded in the \lstinline{jackal_base} node, where it is connected to
Jackal's kinematic controller.

The key topics which comprise Jackal's ROS API are given in \autoref{table:rosapi}.

% TODO: Data flow diagram

\begin{warning}[Incomplete Feature]
Note that due to USB stability problems with Navsat and IMU output enabled, these outputs
have been temporarily disabled in the 0.1 firmware release. When further testing has been
completed, an update will be released to enable these functions.
\end{warning}

\begin{table}[ht]
\begin{tabular}{  l  l  p{7cm} }
\hline
Topic & Message Type & Purpose \\ \hline

\lstinline{/cmd_vel} & \lstinline{geometry_msgs/Twist} & 
Input to Jackal's kinematic controller. Publish here to make Jackal go. \\ \hline
\lstinline{/odometry/filtered} & \lstinline{nav_msgs/Odometry} & 
Published by \lstinline{robot_localization}, a filtered localization estimate based
on wheel odometry (encoders), integrated IMU, and integrated GPS. \\ \hline

\lstinline{/imu/data} & \lstinline{sensor_msgs/IMU} & 
Published by \lstinline{imu_filter_madgwick}, an orientation estimate based on Jackal's
internal gyroscope, accelerometer, and magnetometer. \\ \hline
\lstinline{/navsat/fix} & \lstinline{sensor_msgs/NavSatFix} & 
Position fix from Jackal's built in GPS receiver. \\ \hline
\lstinline{/navsat/vel} & \lstinline{geometry_msgs/TwistStamped} & 
Velocity over ground according to the integrated GPS receiver.\\ \hline

\lstinline{/cmd_drive} & \lstinline{jackal_msgs/Drive} &
Output from Jackal's kinematic controller, input to the motor controllers. Subscribe here for a lower-level look at what's going on. \\ \hline
\lstinline{/feedback} & \lstinline{jackal_msgs/Feedback} &
High-frequency inputs from Jackal's encoders and motor current sensors. \\ \hline
\lstinline{/status} & \lstinline{jackal_msgs/Status} &
Low-frequency status data for Jackal's systems. This information is republished in human
readable form on the \lstinline{diagnostics} topic and is best consumed with the Robot
Monitor. \\ \hline
\end{tabular}
\caption{Jackal ROS API Topics}
\label{table:rosapi}
\end{table}

% TODO: Consider moving this content to ROS wiki?


\begin{comment}
\autoref{arch} is a high-level overview of the ROS
nodes in base Jackal and the data flow through them.

\begin{figure}[hb]
  \centering\footnotesize
  \placeholder{16cm}{18cm}
  \caption{ROS nodes and topics on Jackal.}
  \label{arch}
\end{figure}
\end{comment}

\section{Getting Started}

The first step is to power up your Jackal and have some fun driving it around! If you've
just unpacked Jackal from its shipment packaging, you'll need to open it up and connect the
battery.

Press the power button \raisebox{-0.4em}{\includegraphics[width=0.5cm]{icon-power.pdf}} on
Jackal's HMI panel. The LEDs should show a test pattern, after which you will wait about 30
seconds for the internal PC to finish booting up.

% TODO: Add this content back when the comms LED actually works.
% When the comms LED (\raisebox{-0.4em}{\includegraphics[width=0.5cm]{icon-comms.pdf}}) is
% green, this signals that the PC is finished booting up, and that the PC and MCU are in
% communication. At this point, 
  
Press the PS/P3 button on the Sony Bluetooth controller to sync the controller to Jackal. Once
the small red LED on the controller goes solid, you're paired and ready to drive. Hold the L2 trigger
button, and push the thumbstick forward. For full speed mode, switch to the L1 trigger.

If you're not seeing any action, check \nameref{trouble} on page \pageref{trouble} to
get in touch with support.

\subsection{Wireless Access}

To get Jackal connected to your local wifi, you must first access the internal computer
using a wired connection. Open the chassis, lower the computer tray, and connect to the network port
labeled \lstinline{STATIC} with a standard ethernet cable. Now, set your laptop's ethernet port
to a static IP such as \lstinline{192.168.1.51}, and connect via SSH to
\lstinline{administrator@192.168.1.1}. The default password is \lstinline{clearpath}.

Once connected via wire, execute \lstinline{wicd-curses} to enter the text/curses UI to the
wireless interface configuration daemon (WICD). Within the text UI, you can configure
which wireless network you'd like Jackal to connect to upon system startup.

When the wireless link is established, remove the network cable, re-establish your SSH
session over wireless, and close up Jackal.

\subsection{Remote ROS Connectivity}\label{remote}

Now that Jackal is on the wireless, you can access it via SSH or as a remote ROS master.
Note that in the default configuration, the background job running on Jackal launches with the 
\lstinline{robot_upstart} package, which is configured to set the \lstinline{ROS_IP}
environment variable to the static IP of the \lstinline{em1} ethernet port, by default
\lstinline{192.168.1.1}.

What this means is that in order for a workstation to communicate with Jackal over
wireless, you may need to change one of three things:

\begin{enumerate}
\item If you're confident that Jackal will be operated only where it is connected
to wifi, you could set \lstinline{robot_upstart} to start the background ROS job only
once wifi connects. To change this, run:
\begin{lstlisting}
rosrun robot_upstart install jackal_base/launch/base.launch \
    --job ros --interface wlan0
\end{lstlisting}

\item If you're confident that your network will resolve hostnames correctly, you could
change the generated ROS start script (in \lstinline{/usr/sbin/ros-start}) to set the
\lstinline{ROS_HOSTNAME} env var rather than \lstinline{ROS_IP}.

\item Finally, you can add a static route to your workstation which will route requests
from \lstinline{192.168.1.1} to Jackal's real wireless IP on your network. An example of
this configuration:
\begin{lstlisting}
sudo apt-get install ros-indigo-robot-upstart
export ROS_MASTER_URI=http://192.168.1.1:11311
export ROS_IP=$(rosrun robot_upstart getifip wlan0)
sudo route add -net 192.168.1.1 netmask 255.255.255.255 gw $ROS_IP
\end{lstlisting}
These commands would be executed on your own machine.
\end{enumerate}

Please contact Clearpath Support if guidance is required in selecting and executing a
remote access strategy. For more general details on how ROS works over TCP with
multiple machines, please see:

\url{http://wiki.ros.org/ROS/Tutorials/MultipleMachines}.

For help troubleshooting a multiple machines connectivity issue, see:

\url{http://wiki.ros.org/ROS/NetworkSetup}

\subsection{Jackal Desktop Packages}

To command or observe Jackal from your desktop computer, first set up a basic
ROS installation. See the following page for details:

\url{http://wiki.ros.org/indigo/Installation/Ubuntu}

When your ROS install is set up, install the Jackal desktop packages:

\begin{lstlisting}
sudo apt-get install ros-indigo-jackal-desktop
\end{lstlisting}

Once your remote access to Jackal's ROS master is configured (see options in \autoref{remote}),
you can launch rviz, the standard ROS robot visualization tool:

\begin{lstlisting}
roslaunch jackal_viz view_robot.launch
\end{lstlisting}

From within rviz, you can use interactive markers to drive Jackal, you can visualize its
published localization estimate, and you can visualize any attached sensors which have been
added to its robot description XML (\lstinline{URDF}).

Additionally from the desktop, you can launch the standard RQT Robot Monitor, which
watches the diagnostic output from Jackal's self-montoring capabilities:

\begin{lstlisting}
rosrun rqt_robot_monitor rqt_robot_monitor
\end{lstlisting}

% TODO: Insert images of robot monitor/GUI and rviz window.

\begin{warning}[Incomplete Feature]
Note that in the 0.1 release of the Jackal firmware package, the temperature sensors are not
being published to ROS. The values appearing in the diagnostic output are not reflective of
reality. Please expect a near-term update resolving this.
\end{warning}

\section{Apps}

\begin{warning}[Upcoming Development]
The use of the app launcher to run demonstrations such as from the \lstinline{jackal_navigation}
package is planned, but presently incomplete. The intent is ultimately to package Jackal's demos as
Robot Apps or \textit{rapps}, for ease of launch and getting started.
\end{warning}

When equipped with a laser scanner as is available in the Gazebo simulation, Jackal works with the
standard ROS navigation stack. See \url{http://wiki.ros.org/jackal_navigation}.

A standard outdoor GPS autonomy demonstration using Jackal's built-in sensing is planned, as well
as a calibration app for the internal magnetometer.


\section{Charging \& Battery Maintenance}

Jackal's Li-Ion battery pack may be charged internal to the chassis---simply plug in
the charger to the charge port located under the rear fender. Charging will occur
when Jackal is powered down.

Alternatively, if you have multiple battery packs, you can easily lift the lid and
remove the battery for external charging. When charging externally, remove the pigtail
which adapts the charger to the platform's weather sealed charge port.

The battery pack is manufactured for Clearpath Robotics by AllCell Technologies. The
pack includes integrated protections against fault due to overcurrent, overdischarge,
and short circuit. The pack is rugged and designed for the demanding environments into
which Jackal may be deployed.

However, please take note of the following:

\begin{itemize}
\item The pack must not be stored or operated above \SI{60}{\celsius} or below \SI{-19}{\celsius}.
\item The pack must not be punctured or disassembled.
\item The pack should be returned to Clearpath Robotics for disposal, or to your
local hazardous waste authority.
\item When traveling with Jackal, consult your airline's restrictions regarding lithium
battery packs. If possible, bring the pack in your carry on luggage, where it will
be subject to normal cabin temperatures and pressures.
\end{itemize}

Please contact Clearpath Robotics for additional information about Jackal's battery or
for information about purchasing additional packs.


\section{Payload Integration Guide}

If you're wanting to attach custom hardware to Jackal, you'll have to take care of
mechanical mounting, electrical supply, and software integration. This section
aims to equip you with respect to these challenges.

% TODO: Images throughout this section.

\subsection{Mechanical Mounting}

For external payloads, the recommended configuration is manufacture a metal or plastic bracket
which attaches to the \SI{120}{\mm} square mounting holes supplied in Jackal's lid
panel. The included thumbscrews use an M5 thread, if you wish to replace them with conventional
fasteners. As an alternative to manufacturing a brand new plate, you may remove and modify one
of the included ones.

For rear-facing or back-mounted payloads, it is also possible to replace (or drill into) the
hatch panel which covers over access to Jackal's internal PC.

\subsection{Electrical Integration}\label{payload-elec}

Except for bus-powered USB cameras, most payloads have separate leads for power and data. Data
connections may be brought through the hatch and connected directly to the internal computer. Both
of Jackal's internal computer options support USB3 and Ethernet connectivity. With the performance
PC, the PCIe slot may be used to supply Firewire, Thunderbolt, or additional USB3 ports, as necessary.

Additionally, the internal mounting area may be used for an Ethernet switch, when attaching multiple
Ethernet payloads, or for a PoE power injector as required.

The power lead may likewise be brought through the hatch, and connected to the User Power Board. Pull
out the black terminal block, and use a small screwdriver to securely attach power leads to it.
Confirm voltage and polarity before reconnecting the terminal block.

You may also choose to terminate your payload's power lead with the appropriate crimps and pins
for the four pin Molex connector--- this option may be more convenient if you expect to be adding
and removing your payload from Jackal more frequently and would prefer not to be fiddling with the
terminal block. Contact Clearpath Support for details about these parts.

% TODO: Include a table of the parts and diagram showing the pinout.

\subsection{Software Integration}

ROS has a large ecosystem of sensor drivers, some of which include pre-made URDF descriptions and
even simulation configurations. Please see the following page on the ROS wiki for a partial list:

\url{http://wiki.ros.org/Sensors}

For the best experience, consider purchasing supported accessories from Clearpath Robotics for your
Jackal, which will include simulation, visualization, and driver support. However, we will happily
assist you in integrating your own devices as well.


\section{Contact}\label{trouble}\label{contact}

Clearpath is committed to your success with Jackal. Please get in touch with us and we'll
do our best to get you rolling again quickly: \href{mailto:support@clearpathrobotics.com}{support@clearpathrobotics.com}

To get in touch with a salesperson regarding Jackal or other Clearpath Robotics products, please
email \href{mailto:sales@clearpathrobotics.com}{sales@clearpathrobotics.com}.

If you have a an issue that is specifically about ROS and is something which may be of interest
to the broader community, consider asking it on \href{http://answers.ros.org}{answers.ros.org}.
If you don't get a satisfactory response, please ping us and include a link to your question
as posted there. If appropriate, we'll answer in the ROS Answers context for the benefit of the
community.

Jackal is designed not to require regular maintenance. As it is a newer product, Clearpath
appreciates your patience as we understand its weak-point components and fill out the appropriate
care instructions for the platform.

\end{document}
